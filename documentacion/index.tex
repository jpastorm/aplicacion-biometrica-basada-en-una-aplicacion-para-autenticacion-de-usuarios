
\documentclass[preprint,12pt]{elsarticle}

\usepackage[spanish]{babel}
\usepackage{amssymb}
\usepackage{graphicx}
\usepackage{lineno}
\usepackage[utf8]{inputenc}
\usepackage{url}
\usepackage{natbib} 
\usepackage{amsmath} 
\usepackage{amssymb} 
\usepackage{multicol}

\begin{document}
	
\begin{titlepage}
\centering
	{\includegraphics[width=0.2\textwidth]{./IMAGENES/upt}\par}
	\vspace{1cm}
	{\bfseries\LARGE Universidad Privada de Tacna\par}
	\vspace{1cm}
	{\scshape\Large Escuela Profesional de Ingenieria de Sistemas \par}
	\vspace{2cm}
	{\scshape\LARGE Aplicación biométrica basada en web para la autenticacion de usuarios \par}
	\vspace{2cm}
	{\large Integrantes: Franklin Huichi Contreras, Jose Pastor Mendoza \par}
	\vspace{0.1cm}
	{\large Ciclo: Octavo Ciclo \par}
	\vspace{0.1cm}
	{\large Curso: Seguridad Informatica \par}
	\vspace{0.1cm}
	{\large Docente: Ing. Oscar Jimenez Flores\par}
	\vfill
	{\large Tacna - Perú 2020 \par}

\end{titlepage}

\begin{multicols}{2} 

%% Desarrollo ----------------------------------------------------------------------------------------------------------------
\section{Desarrollo} 
La biometría es una tecnología que puede identificar a una persona en función de sus características físicas. La identificación y el reconocimiento de huellas digitales es un método biométrico que se usa ampliamente en varios tipos de aplicaciones debido a su precisión y confiabilidad. El objetivo principal de este proyecto es analisar un sistema que pueda reconocer si 2 impresiones provienen de la misma persona o no. Con este fin, las imágenes se recopilan primero de un conjunto de datos. Luego, sobre las mismas imágenes, se aplican técnicas de procesamiento de imágenes digitales para mejorar su calidad. Una vez que la imagen se se limpie se encuentra los puntos críticos que luego se comparan según su distancia de Hamming.

\subsection{AAAA} 
	\subsubsection{AAAA} 


%% ----------------------------------------------------------------------------------------------------------------------------------


%% Resultados------------------------------------------------------------------------------------------------------------

\section{Resultados}

%%Ejemplo de cita
\cite{Gartner} 


%% ----------------------------------------------------------------------------------------------------------------------------------


%% CONCLUSIONES ---------------------------------------------------------------------------------------------------------------

\section{Conclusiones}

\begin{itemize}

\item Conclusion 1 : \\

\item Conclusion 2 : \\ 

\item Conclusion 3 : \\ 

\item Conclusion 4 : \\ 
\end{itemize}

%% ----------------------------------------------------------------------------------------------------------------------------------
\end{multicols}

%%  REFERENCIAS BIBLIOGRÁFICAS ------------------------------------------------------------------------------------------
	
	\newpage
	
	\bibliographystyle{apalike} 	%ESTILO
	\bibliography{BIBLIOGRAFIA}	 

\end{document}
